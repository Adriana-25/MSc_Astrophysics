

\section{Mission overview}


\parencite{2016}

\subsection{Scientific objectives}



\subsection{Instrumentation and payload}  
\subsection{Mission design}
\section{Gaia Data Release 3}
\parencite{https://doi.org/10.48550/arxiv.2208.00211}
\section{Variability and Specific Object Studies modules}

%Variability analysis can be defined as the comprehensive assessment of the degree and character of patterns of variation over time intervals.

As stated in \cite{Rimoldini}, time-dependent brightness variations of celestial objects may
be caused by different phenomena, and a certain set of
classes can be identified to describe different variability types.
Indeed Gaia offers the unique opportunity to study variability of close to 2 billion objects: its
%thanks to the multiband G, G BP , G RP photometric time series, but also to the other spectrophotometric and RVS time series.
multi-epoch observations and sparse sampling allow for the
detection of periodic signals ranging from minutes to years and
for medium to long-term non-periodic variability.
%The Gaia DR3 photometric time series provided sufficient information to classify ten million variable objects into two dozen variability class groups across the whole sky
%\parencite{Rimoldini}
% for this variability analysis: astrometry, photometry, spectroscopy (epoch radial velocity). I also used some astrophysical parameters

Among the strong points of the mission for variability analysis, we can mention, in addition to its well-known astrometric
capabilities, the large dynamical range reached in stellar brightness, from a few magnitudes to fainter than 20 mag, the specific scanning law leading to irregularly sampled time series,
and the quasi-simultaneity (within tens of seconds) of the observations in G photometry, $G_{BP}$ and $G_{RP}$ spectrophotometry, and
RVS (Radial Velocity Spectrometer) spectroscopy.





\begin{comment}
filtered magnitude time series in G FoV (i.e. averaging the G CCD
measurements within one FoV transit), G BP , and G RP as
described in Eyer et al. (2017) and in Holl et al. (2018)
\end{comment}


% flux variations, variations in brightness
% The structure function characterizes light curve variability by quantifying the change in amplitude Δmij as a function of time lag Δtij between observations at epochs i and j. Following the prescription of Schmidt et al. (2010), variability structure function of the source magnitude
% Parameters γ and Ar of the variability structure function for the stellar (blue points) and quasar (red points) test samples. Large A’s indicate large fluctuation amplitudes. Large γ’s indicate an increase of the fluctuation amplitude with time.
% method that characterizes light curve variability

\section{Gaia Roadmap and legacy}



